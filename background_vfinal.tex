\documentclass{article}

\usepackage{amsmath, amssymb, graphicx}
\usepackage{physics}
\usepackage{cite}
\usepackage{natbib}

\title{Background: Viscous Motion Via NVE Molecular Dynamics Simulation}
\author{Jorge R. Gherson}
\date{\today}

\begin{document}
\maketitle

\section{Introduction}
Molecular dynamics (MD) simulations provide a powerful computational framework to study the thermodynamic and structural properties of 
condensed matter systems, including supercooled liquids and metallic glasses. By numerically integrating Newton's equations of motion, 
MD tracks the time evolution of atomic trajectories and thereby reveals microscopic mechanisms underlying phase transitions 
\cite{allen1989computer, frenkel2002understanding}. 

Among various computational models for glass formation, the  \textbf{Kob - Andersen (KA) model} has proven especially valuable for 
investigating the dynamics of supercooled liquids. Designed as a binary Lennard-Jones (LJ) system, 
the KA model suppresses crystallization through asymmetrical 
interaction parameters, facilitating the formation of stable amorphous 
states \cite{kob1994scaling, kob1995testing}.

\section{Molecular Dynamics Formulation}
\subsection{Newton's Equations of Motion}
The motion of particles in a molecular system is governed by Newton’s second law, which states that the force acting on a particle is 
equal to its mass multiplied by its acceleration. In mathematical terms, for a system of \(N\) particles, each having mass \(m_i\), 
and position \(\mathbf{r}_i\), the equation takes the form:
\begin{equation}
    \mathbf{F}_i = m_i \frac{d^2 \mathbf{r}_i}{dt^2},
\end{equation}

The force \(\mathbf{F}_i\) is derived from the interatomic potential \(U(\mathbf{r})\) by taking its negative gradient:
\begin{equation} \label{eq:force}
    \mathbf{F}_i = -\mathbf{\nabla}_i \, U(\mathbf{r}_i).
\end{equation}

Thus, acceleration is given by:
\begin{equation} \label{eq:acceleration}
    \mathbf{a}_i = \frac{\mathbf{F}_i}{m_i} =\frac{1}{m_i}\begin{bmatrix}
        \partial_{x_i}\sum_j U_{ij} \\
        \partial_{y_i}\sum_j U_{ij} \\
        \partial_{z_i}\sum_{j} U_{ij}
    \end{bmatrix}
\end{equation}

\subsection{The Lennard-Jones Potential}
A fundamental model for describing interatomic interactions is the Lennard-Jones potential, which captures the balance between 
attractive and repulsive forces \cite{vollmayr2020introduction}. This potential is mathematically represented as:
\begin{equation}\label{eq:LJ}
    U(r) = 4\epsilon \left[ \left(\frac{\sigma}{r}\right)^{12} - \left(\frac{\sigma}{r}\right)^{6} \right],
\end{equation}
where \(\epsilon\) represents the depth of the potential well (i.e., the strength of the attractive interaction),
\(\sigma\) is approximately the distance at which the interparticle potential becomes zero,
the \(r^{-12}\) term models the steep repulsion due to overlapping electron orbitals (Pauli exclusion principle),
and finally, the \(r^{-6}\) term captures the long-range van der Waals attractions.

\subsection{Kob-Andersen Potential}
The Kob-Andersen potential modifies the Lennard-Jones potential by allowing the parameters \(\epsilon\) and \(\sigma\) to depend on the particle types \(\alpha,\beta \in \{A,B\}\) of 
particles \(i\) and \(j\) for an \(80:20 \, A:B\) mixture \cite{vollmayr2020introduction}. This adjustment can be used to capture additional anharmonic effects in atomic vibrations. The potential is given by:
\begin{equation}\label{eq:KA}
    U_{ij} = U_{\alpha\beta}(r_{ij}) = 4\epsilon_{\alpha\beta} \left[ \left(\frac{\sigma_{\alpha\beta}}{r_{ij}}\right)^{12} - \left(\frac{\sigma_{\alpha\beta}}{r_{ij}}\right)^{6} \right]
\end{equation}
with the understanding that in some implementations, an additional Morse potential term may be included to further capture anharmonic effects \cite{kob1994scaling}.
Here, we use the same units as in \cite{vollmayr2020introduction} such that \(\sigma_{AA} = 1\) (length unit), \(\epsilon_{AA}=1\) (energy unit), \(m_A = 1\) (mass unit), and 
\(k_B=1\) for the temperature unit \(\epsilon_{AA}/k_b\). The time unit is \(\sqrt{m_A \sigma^2_{AA}/\epsilon_{AA}}\).
This results in our KA parameters being \(\sigma_{AA}=1,\, \epsilon_{AA}=1,\, \sigma_{AB}=0.8,\, \epsilon_{AB}=1.5,\, \sigma_{BB}=0.88,\, \epsilon_{BB}=0.5\), and 
\(m_A = m_B = 1.0\) \cite{vollmayr2020introduction, kob1994scaling}.

\subsection{Explicit Calculation of Total Potential Energy \(U_{\text{tot}}\)}
The total potential energy of the system is obtained by summing over all unique pairwise interactions:
\begin{equation} \label{eq:Utot}
    U_{\text{tot}} = \sum_{i<j} U_{ij} = \sum_{i<j} 4\epsilon_{\alpha\beta} \left[ \left(\frac{\sigma_{\alpha\beta}}{r_{ij}}\right)^{12} - \left(\frac{\sigma_{\alpha\beta}}{r_{ij}}\right)^{6} \right]
\end{equation}
Here, the summation \(i<j\) ensures that each pair interaction is counted only once.
The force between particles \(i\) and \(j\) is derived from \(U_{ij}\) by differentiation:
\begin{equation}
    \mathbf{F}_{ij} = -\frac{d U_{ij}}{d r_{ij}} \hat{\mathbf{r}}_{ij}
\end{equation}
where \( \hat{\mathbf{r}}_{ij} = \frac{\mathbf{r}_i - \mathbf{r}_j}{r_{ij}} \). Differentiating Equation \eqref{eq:KA} with respect to \(r_{ij}\) gives:
\begin{equation}
    \frac{d U_{ij}}{d r_{ij}} = 4\epsilon_{\alpha\beta} \left[-12 \left(\frac{\sigma_{\alpha\beta}}{r_{ij}}\right)^{12} \frac{1}{r_{ij}} + 6 \left(\frac{\sigma_{\alpha\beta}}{r_{ij}}\right)^{6} \frac{1}{r_{ij}}\right]
\end{equation}
Thus,
\begin{equation}
    F_{ij} = 24 \epsilon_{\alpha\beta} \left[\frac{2\sigma_{\alpha\beta}^{12}}{r_{ij}^{13}} - \frac{\sigma_{\alpha\beta}^{6}}{r_{ij}^{7}} \right]
\end{equation}
and the total force on particle \(i\) is:
\begin{equation}
    \mathbf{F}_i = \sum_{j \neq i} \mathbf{F}_{ij} = \sum_{j \neq i} 24 \epsilon_{\alpha\beta} \left[\frac{2\sigma_{\alpha\beta}^{12}}{r_{ij}^{13}} - \frac{\sigma_{\alpha\beta}^{6}}{r_{ij}^{7}} \right] \hat{\mathbf{r}}_{ij}
\end{equation}

\subsection{Periodic Boundary Conditions and the Minimum Image Convention}
In a finite simulation, Periodic Boundary Conditions (PBCs) are used to mimic an infinite system. When a particle moves out of the simulation box, it re-enters from the opposite side, thereby eliminating surface effects.

The Minimum Image Convention ensures that the interparticle distance \(r_{ij}\) is computed as the shortest distance between a particle and any periodic image of another particle. Mathematically, if the simulation box has length \(L\) in each dimension, then the distance between particles \(i\) and \(j\) is given by:
\begin{equation}
    r_{ij} = \min_{\mathbf{n}} |\mathbf{r}_j - \mathbf{r}_i - L\mathbf{n}|
\end{equation}
where \(\mathbf{n} = (n_x, n_y, n_z)\) with \( n_x, n_y, n_z \in \{-1,0,1\}\). This procedure ensures that interactions are computed using the closest image of each particle.

A \textbf{cutoff radius} \(r_{\text{cut}}\) is applied to further reduce computational expense by considering interactions only if:
\[
\textbf{r}_{ij} < \textbf{r}_{\text{cut}}
\]
typically chosen as \(r_{\text{cut}} \approx 2.5\sigma\), where \(U_{ij}(\textbf{r}_{ij})=0\) 
for \(\textbf{r}_{ij}\geq \textbf{r}_{\text{cut}}\) (i.e., the Lennard Jones Potential \(\alpha\beta\) becomes negligible) \cite{vollmayr2020introduction,kob1994scaling}. 
To avoid a discontinuity in the potential energy, a \textbf{cutoff function} is introduced to smoothly truncate the potential at \(r_{\text{cut}}\).
So, we can define 
\begin{equation}
    U_{ij}^{\text{cut}} = 
    \begin{cases}
        U_{\alpha\beta}(r_{ij}) - U_{\alpha\beta}(r_{\alpha\beta}^{\text{cut}}) & r_{ij} < r_{\alpha\beta}^\text{cut} \\
        0 & \text{otherwise}
    \end{cases}
\end{equation}

\subsection{Temperature Initialization}
Initial velocities are sampled from the Maxwell--Boltzmann distribution to set 
the system’s temperature. In practice, each velocity component is drawn from a 
Gaussian distribution with zero mean and variance \(\sqrt{\frac{k_B T}{m_i}}\). 
This approach ensures that the overall velocity distribution corresponds to the 
desired temperature, allowing the simulation to proceed in an NVE ensemble using 
the Velocity Verlet algorithm without an active thermostat.

\section{Simulation: Integration \& Analysis}
\subsection{Integration via the Velocity Verlet Algorithm}
To update the positions and velocities, the Velocity Verlet algorithm is employed. The equations are:
\begin{align}
    \mathbf{r}_i(t+\Delta t) &= \mathbf{r}_i(t) + \mathbf{v}_i(t)\Delta t + \frac{1}{2} \mathbf{a}_i(t)\Delta t^2 \label{eq:verlet1} \\
    \mathbf{v}_i(t+\Delta t) &= \mathbf{v}_i(t) + \frac{1}{2}\left[\mathbf{a}_i(t) + \mathbf{a}_i(t+\Delta t)\right]\Delta t \label{eq:verlet2}
\end{align}\cite{allen1989computer}.
Equation \eqref{eq:verlet1} updates the positions using the current velocities 
and accelerations, while Equation \eqref{eq:verlet2} refines the velocities using
the average of the current and new accelerations. This integration scheme is 
both symplectic and time-reversible, making it ideal for energy conservation over
long simulation times.

\subsection{Specific Heat Capacity}
The thermodynamic behavior of a system can be characterized by its specific heat capacity at constant volume, \(C_V\). This quantity 
can be computed from fluctuations in the system’s total energy \cite{mcquarrie2000statistical}:
\begin{equation}\label{eqhc}
    C_V = \frac{1}{k_B T^2} \left( \langle E^2 \rangle - \langle E \rangle^2 \right)
\end{equation}
where \(E\) is the total energy of the system, \(k_B\) is the Boltzmann constant, and \(T\) is the absolute temperature. 
The term \(\langle E^2 \rangle - \langle E \rangle^2\) represents the variance in energy, which increases as the system approaches a 
phase transition.

\subsection{Radial Distribution Function and Structure Factor}
To gain insight into phase transitions, the radial distribution function (RDF), \(g(r)\), is used to describe the probability of 
finding a particle at a distance \(r\) from a reference particle \cite{allen1989computer}. 
It is a measure of the density of particles \(j\) at a distance \(r\) from a reference particle \(i\),
where \(r=r_{ij} = |\bf{r}_i-\bf{r}_j|\) and radial symmetry is assumed \cite{vollmayr2020introduction}. 
For a binary system, the RDF is defined as:
\begin{equation}\label{eqrdf}
    g_{\alpha\alpha}(r) = \frac{V}{N_{\alpha}(N_\alpha-1)} \langle\sum_{i=1}^{N_\alpha}\sum_{j=1, j\neq i}^{N_\alpha}\delta(r - |\bf{r}_i-\bf{r}_j|)\rangle
\end{equation}
where \(V\) is the system volume, \(N\) is the number of particles, and \(r_{ij}\) is the distance between particles \(i\) and \(j\) \cite{kob1995testing}. 
In numerical simulations, \(g(r)\) is computed by binning interparticle distances and normalizing the counts. 
The Dirac delta function \(\delta(r - |\bf{r}_i-\bf{r}_j|)\) ensures that only particles at a distance \(r\) contribute to the sum. 
Therefore, the RDF provides information about the system’s structure, such as the presence of short-range order in solids or the absence of long-range order in liquids \cite{vollmayr2020introduction}.

\bibliographystyle{unsrt}
\bibliography{references}

\end{document}