\documentclass{beamer}
\usetheme{Boadilla} % Change theme if desired
\usepackage[utf8]{inputenc}
\usepackage{graphicx}
\usepackage{amsmath}
\usepackage{multicol}

\title{Evaluating Glass Forming Compounds via NVE Ensemble Molecular Dynamics Simulation}
\subtitle{Background and Methodology for PHYS338 Final Project}
\author{Jorge R. Gherson}
\date{March 24, 2025}

\begin{document}

% Title slide
\begin{frame}
  \titlepage
\end{frame}

% Motivation slide
\section{Motivation}
\begin{frame}{Motivation and Project Goals}
  \begin{itemize}
    \item \textbf{Why study glass formers?} They exhibit complex, non-crystalline behavior and unique energy dynamics that are critical to understanding material properties.
    \item \textbf{Role of MD:} Molecular Dynamics (MD) simulations allow us to explore microscopic structural and dynamic phenomena.
    \item \textbf{Project Aim:} Use an NVE ensemble to preserve energy conservation while analyzing diffusion and structure in glass forming compounds.
  \end{itemize}
\end{frame}

% Section 1: MD Formulation
\section{MD Formulation}
\begin{frame}{Molecular Dynamics Formulation}
  \begin{itemize}
    \item From Newton's second law:
      \[
      \mathbf{F}_i = m_i \frac{d^2 \mathbf{r}_i}{dt^2}.
      \]
    \item The force is obtained from the interatomic potential \(U(\mathbf{r})\):
      \[
      \mathbf{F}_i = -\nabla_i U(\mathbf{r}_i).
      \]
  \end{itemize}
\end{frame}

% Section 2: Interatomic Potentials
\section{Interatomic Potentials}
\begin{frame}{Lennard-Jones Potential}
  \begin{itemize}
    \item A simple yet widely used model for interatomic interactions:
      \[
      U(r_{ij}) = 4\epsilon \left[\left(\frac{\sigma}{r_{ij}}\right)^{12} - \left(\frac{\sigma}{r_{ij}}\right)^6\right].
      \]
    where:
    \begin{itemize}
      \item \(\epsilon\) is the depth of the potential well.
      \item \(\sigma\) is the finite distance at which the inter-particle potential is zero.
      \item \(r_{ij}\) is the distance between particles \(i\) and \(j\).
      \item The equilibrium separation is at \(r_{ij}=2^{1/6}\sigma\).
    \end{itemize}
  \end{itemize}
\end{frame}

\begin{frame}{Kob-Andersen Model}
    \begin{itemize}
      \item An extension of the Lennard-Jones model tailored for binary mixtures.
      \item For particle types \(\alpha,\beta \in \{A,B\}\) in an 80:20 A:B mixture:
        \[
        U_{ij} = U_{\alpha\beta}(r_{ij}) = 4\epsilon_{\alpha\beta}\left[\left(\frac{\sigma_{\alpha\beta}}{r_{ij}}\right)^{12} - \left(\frac{\sigma_{\alpha\beta}}{r_{ij}}\right)^6\right].
        \]
      \item This adjustment captures additional anharmonic effects relevant to glass formation.
    \end{itemize}
\end{frame}

% Section 3: Simulation Method
\section{Simulation Method}
\begin{frame}{Velocity-Verlet Algorithm}
  To update positions and velocities, we use the Velocity-Verlet algorithm:
  \begin{align}
    \mathbf{r}_i(t+\Delta t) &= \mathbf{r}_i(t) + \mathbf{v}_i(t)\Delta t + \frac{1}{2} \mathbf{a}_i(t)\Delta t^2, \label{eq:verlet1} \\
    \mathbf{v}_i(t+\Delta t) &= \mathbf{v}_i(t) + \frac{1}{2}\left[\mathbf{a}_i(t) + \mathbf{a}_i(t+\Delta t)\right]\Delta t. \label{eq:verlet2}
  \end{align}
  \vspace{0.5em}
  \begin{itemize}
    \item \textbf{Benefits:} Time-reversible, stable, and energy conserving.
    \item \textbf{Ensemble:} NVE ensures constant total energy during simulation.
  \end{itemize}
\end{frame}

\begin{frame}{Periodic Boundary Conditions}
  \begin{multicols}{2}
  \begin{itemize}
    \item \textbf{Purpose:} Mimic an infinite system by wrapping particles into a simulation box.
    \item \textbf{Minimum Image Convention:}
      \[
      r_{ij} = \min_{n \in \{-1,0,1\}^3} \left| \mathbf{r}_j - \mathbf{r}_i - \mathbf{L}\cdot\mathbf{n} \right|
      \]
    \item A cutoff radius \(r_{cut} \approx 2.5\sigma\) is applied to reduce computation.
    \item To avoid discontinuities, define \(U^{\text{cut}}_{ij}\) at \(r_{cut}\): 
    \vspace{0.2cm}
    \begin{equation*}
      U_{ij}^{\text{cut}} = 
      \begin{cases}
          U_{\alpha\beta}(r_{ij}) - U_{\alpha\beta}(r_{\alpha\beta}^{\text{cut}}) & r_{ij} < r_{\alpha\beta}^\text{cut} \\
          0 & \text{otherwise}
      \end{cases}
    \end{equation*}
  \end{itemize}
  \columnbreak
  \includegraphics[width=\linewidth]{./figures/fig1.png} % Placeholder: Insert schematic diagram here.
  \end{multicols}
\end{frame}

% Section 4: Analysis Techniques
\section{Analysis Techniques}
\begin{frame}{Analysis Techniques}
  \begin{itemize}
    \item \textbf{Radial Distribution Function (RDF):} Measures local structural order and describes the probability of 
    finding a particle at a distance \(r\) from another reference particle.
      \[
      g_{\alpha\alpha}(r) = \frac{V}{N_\alpha (N_\alpha-1)} \Bigg\langle \sum_{i=1}^{N_\alpha} \sum_{\substack{j=1 \\ j\neq i}}^{N_\alpha} \delta\Big(r-|\mathbf{r}_i-\mathbf{r}_j|\Big) \Bigg\rangle.
      \] 
    where the Dirac delta function \(\delta(r - |\bf{r}_i-\bf{r}_j|)\) ensures that only particles at a distance \(r\) contribute to the sum.
    \item \textbf{Mean Square Displacement (MSD):} Quantifies diffusion by tracking particle displacement over time.
  \end{itemize}
\end{frame}

% Section 5: Conclusions & Future Work
\section{Conclusions \& Future Work}
\begin{frame}{Conclusions \& Future Directions}
  \begin{itemize}
    \item MD simulations provide valuable microscopic insight into glass forming compounds.
    \item The NVE ensemble effectively conserves energy, making it ideal for studying diffusion and energy dynamics.
    \item \textbf{Future Directions:}
      \begin{itemize}
        \item Incorporate advanced analysis techniques to further probe dynamic properties. 
        \item Explore different binary mixtures to understand their impact on glass formation.
      \end{itemize}
  \end{itemize}
\end{frame}

\end{document}
